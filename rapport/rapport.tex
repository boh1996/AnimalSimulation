\documentclass[a4paper]{article}

%%%%%%%%%%%%%%%%%%%%%%%%%%%%%%%%%% 
% Package for making LaTeX properly handle utf8 characters set and danish language rules
\usepackage[utf8]{inputenc}
\usepackage[danish]{babel}

%%%%%%%%%%%%%%%%%%%%%%%%%%%%%%%%%% 
% Package for changing to a nicer font 
\usepackage[T1]{fontenc}

%%%%%%%%%%%%%%%%%%%%%%%%%%%%%%%%%% 
% Package for conctroling the text area
\usepackage[margin=2.5cm]{geometry}

%%%%%%%%%%%%%%%%%%%%%%%%%%%%%%%%%% 
% Package for inserting clickable hyperlinks in pdf versions as produced by pdflatex
\usepackage{enumitem, hyperref}

%%%%%%%%%%%%%%%%%%%%%%%%%%%%%%%%%% 
% Package for including figures. TeX and thus LaTeX was developped before the existence of directory file-structures, but the graphicspath let's you add directories, that the \includegraphics will search.
\usepackage{graphicx}
\graphicspath{{figures/}}

%%%%%%%%%%%%%%%%%%%%%%%%%%%%%%%%%% 
% Package for typesetting programs. Listings does not support fsharp, but a little modification goes a long way
\usepackage{listings}
\usepackage{xcolor}
\usepackage{verbatim}
\usepackage{color}
\usepackage{textcomp}

\renewcommand{\figurename}{Figur}
\renewcommand{\contentsname}{Indholdsfortegnelse}
\renewcommand{\contentsname}{Table of Contents}
\renewcommand{\lstlistingname}{Kildekode}
\renewcommand{\partname}{Afsnit}

\def\sectionautorefname~#1\null{%
  Afsnit #1\null
}

\def\subsectionautorefname~#1\null{%
  Afsnit #1\null
}

\def\figureautorefname~#1\null{%
  Figur #1\null
}

\def\equationautorefname~#1\null{%
  Ligning #1\null
}

\def\namedlabel#1#2{\begingroup
    #2%
    \def\@currentlabel{#2}%
    \phantomsection\label{#1}\endgroup
}

\definecolor{bluekeywords}{rgb}{0.13,0.13,1}
\definecolor{greencomments}{rgb}{0,0.5,0}
\definecolor{turqusnumbers}{rgb}{0.17,0.57,0.69}
\definecolor{redstrings}{rgb}{0.5,0,0}
\definecolor{lightgray}{RGB}{240, 240, 240}

\newcommand{\namedref}[1]{\autoref{#1} - \nameref{#1} på Side \pageref{#1}}

\lstdefinelanguage{FSharp}
				{morekeywords={\#load, \#r, let, new, match, with, rec, open, module, namespace, type, of, member, and, for, in, do, begin, end, fun, function, try, mutable, if, then, else, List, Set, Sudoku, Seq, false, true, Assert, printfn, print, sprintf, when, ->, >, ::, printf, yield, this},
	keywordstyle=\color{bluekeywords},
	sensitive=false,
	morecomment=[l][\color{greencomments}]{///},
	morecomment=[l][\color{greencomments}]{//},
	morecomment=[s][\color{greencomments}]{{(*}{*)}},
	morestring=[b]",
	stringstyle=\color{redstrings},
	tabsize=2, % sets default tabsize to 2 spaces
	backgroundcolor=\color{lightgray}
}

\usepackage[table]{}
\usepackage{array}
\usepackage{algorithm}
\usepackage{caption}
\usepackage{float}
\usepackage{amsmath}
\usepackage{algorithm}
\usepackage[noend]{algpseudocode}
\usepackage{mathtools}
\usepackage{ragged2e}
\usepackage{caption}
\usepackage{amssymb}
\usepackage{listingsutf8}
\usepackage{newunicodechar}
\usepackage{nameref}
%\newunicodechar{┌}{?}

% Site med mange eksempler - både kode og billeder
% http://www.texample.net/tikz/

%%% Tikz: pakke til at lave grafik
\usepackage{tikz}

\usepackage{url}

%%% Tilføjer makroer der gør livet lettere for os
\usetikzlibrary{arrows, shapes, positioning}

%%% Pile
%%% http://www.texample.net/media/pgf/builds/pgfmanualCVS2012-11-04.pdf (afsnit 24)
\tikzset{
  aggr/.style={->, >=open diamond},
  inh/.style={->, >=open triangle 45}
} 

%%% Bokse
%%% http://www.texample.net/media/pgf/builds/pgfmanualCVS2012-11-04.pdf (afsnit 49)
\tikzset{
  class/.style={draw, rectangle}
}

\lstset{ %
  numbers=right,                    % where to put the line-numbers; possible values are (none, left, right)
  numbersep=5pt,                   % how far the line-numbers are from the code
  numberstyle=\small\color{bluekeywords}, % the style that is used for the line-numbers
  stepnumber=1,                    % the step between two line-numbers. If it's 1, each line will be numbered
  title=\lstname,                   % show the filename of files included with \lstinputlisting; also try caption instead of title
  showstringspaces=false,
  breaklines=true,
  captionpos=b,
  language=FSharp,
  texcl=true,
  inputencoding=utf8,
  extendedchars=true,
  mathescape=true,
  %escapeinside={(*}{*)},
  extendedchars=\true,
  inputencoding=ansinew
  }

\setlength\parindent{0pt}

%%%%%%%%%%%%%%%%%%%%%%%%%%%%%%%%%%
% Package or using suits
\usepackage{kmath,kerkis}
\normalfont

\usepackage{fancyhdr}
\usepackage{lastpage}
 
\pagestyle{fancy}
\fancyhf{}
 
\rhead{Mads U. Svendsen, Anders F. Jørgensen, Nicolai L. Hargreave, Bo H. Thomsen}
\rfoot{Side \thepage \hspace{1pt} af \pageref{LastPage}}

\title{Prey \& The Predators - Programmering og Problemløsning}
\author{Mads U. Svendsen, Anders F. Jørgensen, Nicolai L. Hargreave, Bo H. Thomsen}

\begin{document}
	\maketitle
  \tableofcontents
\section{Forord}
  Denne opgave er lavet i Programmering og Problemløsning(PoP),
    på Datalogisk Institut - Københavns Universitet(DIKU) - første semester år 2015/2016.
    Opgaven har opgavenummeret 11g.
    I gruppen er Mads U. Svendsen, Anders F. Jørgensen, Nicolai L. Hargreave og Bo H. Thomsen.
\newpage
    
\section{Introduktion} \label{sec:introduction}
   I dette projekt er målet at udvikle en simulator,
   den kan simulere det kendte Prey/Predator udviklings problem.
   Simulatoren skal vise udviklingen blandt en gruppe af byttedur, 
   og en gruppe af rovdyr. Hvordan mængden af rovdyr i perioder stiger,
   men når de så når et specielt sted - falder kraftigt. Hvorefter mængden af byttedyr stiger.
   Målet for projektet er at vende sig til objekter, klasser, abstrakte klasser og inheritance.

  \paragraph*{Sådan kompilerer du projektet\\}
    I \path{/src} mappen ligger \path{simulation.fsx},
    den kan kompileres med fsharpc og køres med mono.
    Den generere \path{simulation.exe}.

    Tests findes i  mappen \path{/src}, i filen \path{tests.fsx}.

\section{Problemformulering} \label{sec:problem}
  Vi vil i dette projekt, bygge en ssimulator der kan simulere,
  hvordan flere arter lever sammen i et hapitat.
  Når nogle typer har den egenskab, at de kan spise andre.
  I dette eksempel operere vi kun med to typer dyr, "Prey" og "Predator".


  \subsection{Kravspecifikation} \label{ssec:demands}
    
  \section{Problemanalyse og design} \label{sec:design}
    

  \section{Programbeskrivelse} \label{sec:programDescription}
    
      
  \section{Afprøvning} \label{sec:unitTest}
    

	\section{Diskussion og konlusion} \label{sec:conclusion}
   
  \section{Bilag}
     
      \subsection{Brugervejledning} \label{ssec:manual}
        
      \subsection{Kildekode} \label{ssec:sourceCode}
        
        \lstinputlisting[caption={Simulations klasser},label={lst:classes}]{../src/classes.fsx}
        \lstinputlisting[caption={Logik},label={lst:simulation}]{../src/simulation.fsx}
        \lstinputlisting[caption={Tests},label={lst:tests}]{../src/test.fsx}

      \subsection{Billeder}
        
\end{document}